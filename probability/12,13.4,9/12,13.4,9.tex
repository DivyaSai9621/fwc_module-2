\def\mytitle{PROBABILITY}
\def\myauthor{Divya Sai}
\def\contact{nanneboinadivyasai@gmail.com}
\def\mymodule{Future Wireless Communication (FWC)}
\documentclass[10pt, a4paper]{article}
\usepackage[a4paper,outer=1.5cm,inner=1.5cm,top=1.75cm,bottom=1.5cm]{geometry}
\usepackage{graphicx}
\graphicspath{{./images/}}
\usepackage[colorlinks,linkcolor={black},citecolor={blue!80!black},urlcolor={blue!80!black}]{hyperref}
\usepackage[parfill]{parskip}
\usepackage{lmodern}
\usepackage{tikz}
\usepackage{tabularx}
\usepackage{tensor}
\usepackage{amsmath}
\newcommand\Perms[2]{\tensor[^{#2}]P{_{#1}}}
\usepackage{circuitikz}
\usetikzlibrary{calc}
\usepackage{amsmath}
\usepackage{amssymb}
\renewcommand*\familydefault{\sfdefault}
\usepackage{watermark}
\usepackage{lipsum}
\usepackage{xcolor}
\usepackage{listings}
\usepackage{float}
\usepackage{titlesec}
\newcommand{\myvec}[1]{\ensuremath{\begin{pmatrix}#1\end{pmatrix}}}
\newcommand{\mydet}[1]{\ensuremath{\begin{vmatrix}#1\end{vmatrix}}}
\providecommand{\brak}[1]{\ensuremath{\left(#1\right)}}
\providecommand{\lbrak}[1]{\ensuremath{\left(#1\right.}}
\providecommand{\rbrak}[1]{\ensuremath{\left.#1\right)}}
\providecommand{\sbrak}[1]{\ensuremath{{}\left[#1\right]}}
\providecommand{\mtx}[1]{\mathbf{#1}}
\titlespacing{\subsection}{1pt}{\parskip}{3pt}
\titlespacing{\subsubsection}{0pt}{\parskip}{-\parskip}
\titlespacing{\paragraph}{0pt}{\parskip}{\parskip}
\newcommand{\figuremacro}[5]{
    \begin{figure}[#1]
        \centering
   
        \includegraphics[width=#5\columnwidth]{#2}
        \caption[#3]{\textbf{#3}#4}
        \label{fig:#2}
    \end{figure}
}
\let\vec\mathbf
\lstset{
frame=single, 
breaklines=true,
columns=fullflexible
}

\title{\mytitle}
\author{\myauthor\hspace{1em} \\\contact\\FWC22094\hspace{6.5em}IITH\hspace{0.5em}\mymodule\hspace{6em}Module 2}
\date{}
\begin{document}
\maketitle
\paragraph*{\large Q-12,13.4,9}
\paragraph*{\large The random variable X has a probability distribution P(X) of the following form.where k is some number: }
\hspace{15mm}\\
a) $P(A|B) = \dfrac{P(B)}{P(A)}\hspace{20mm}$ b)$P(A|B) < P(A)$
\\
\\
c)$P(A|B) \geqq P(A)$ \hspace{21mm} d)None of these\\
\\
\noindent \textbf{Given:} A $\subset$ B and P(B) $\neq$ 0\\
\textbf{solution}\\
if A $\subset$ B and P(B)$\neq$ 0 then \\
\begin{center}
$\Rightarrow$ A $\cap$ B =A\\
also P(A) $<$ P(B)\\\vspace{2mm}
\end{center}
\begin{equation}
P(A|B) = \dfrac{P(A \cap B)}{P(B)}=\dfrac{P(A)}{P(B)}\\
\end{equation}
we know that \\
\begin{center}
1 $\leq$ $\dfrac{1}{P(B)}$
\end{center}
multiply both sides with P(A),we get
\begin{center}
P(A) $\leq$ $\dfrac{P(A)}{P(B)}$
\end{center}
from the above eq 10
\begin{center}
P(A) $\leq$ P(A$\mid$B)
\end{center}
\begin{equation}
\boxed{ P(A \mid B)\geq P(A)}
\end{equation}
\end{document}