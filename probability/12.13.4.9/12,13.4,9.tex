\def\mytitle{PROBABILITY}
\def\myauthor{Divya Sai}
\def\contact{nanneboinadivyasai@gmail.com}
\def\mymodule{Future Wireless Communication (FWC)}
\documentclass[10pt, a4paper]{article}
\usepackage[a4paper,outer=1.5cm,inner=1.5cm,top=1.75cm,bottom=1.5cm]{geometry}
\usepackage{graphicx}
\graphicspath{{./images/}}
\usepackage[colorlinks,linkcolor={black},citecolor={blue!80!black},urlcolor={blue!80!black}]{hyperref}
\usepackage[parfill]{parskip}
\usepackage{lmodern}
\usepackage{tikz}
\usepackage{tabularx}
\usepackage{tensor}
\usepackage{amsmath}
\newcommand\Perms[2]{\tensor[^{#2}]P{_{#1}}}
\usepackage{circuitikz}
\usetikzlibrary{calc}
\usepackage{amsmath}
\usepackage{amssymb}
\renewcommand*\familydefault{\sfdefault}
\usepackage{watermark}
\usepackage{lipsum}
\usepackage{xcolor}
\usepackage{listings}
\usepackage{float}
\usepackage{titlesec}
\newcommand{\myvec}[1]{\ensuremath{\begin{pmatrix}#1\end{pmatrix}}}
\newcommand{\mydet}[1]{\ensuremath{\begin{vmatrix}#1\end{vmatrix}}}
\providecommand{\brak}[1]{\ensuremath{\left(#1\right)}}
\providecommand{\lbrak}[1]{\ensuremath{\left(#1\right.}}
\providecommand{\rbrak}[1]{\ensuremath{\left.#1\right)}}
\providecommand{\sbrak}[1]{\ensuremath{{}\left[#1\right]}}
\providecommand{\mtx}[1]{\mathbf{#1}}
\titlespacing{\subsection}{1pt}{\parskip}{3pt}
\titlespacing{\subsubsection}{0pt}{\parskip}{-\parskip}
\titlespacing{\paragraph}{0pt}{\parskip}{\parskip}
\newcommand{\figuremacro}[5]{
    \begin{figure}[#1]
        \centering
   
        \includegraphics[width=#5\columnwidth]{#2}
        \caption[#3]{\textbf{#3}#4}
        \label{fig:#2}
    \end{figure}
}
\let\vec\mathbf
\lstset{
frame=single, 
breaklines=true,
columns=fullflexible
}

\title{\mytitle}
\author{\myauthor\hspace{1em} \\\contact\\FWC22094\hspace{6.5em}IITH\hspace{0.5em}\mymodule\hspace{6em}Module 2}
\date{}
\begin{document}
\maketitle
\paragraph*{\large Q-12,13.4,9}
\paragraph*{\large The random variable X has a probability distribution P(X) of the following form.where k is some number: }
\begin{equation}
  P(X) =
    \begin{cases}
      k,  & \text{if x=0}\\
      2k, & \text{if x=1}\\
      3k, & \text{if x=2}\\
      0 , & \text{otherwise}
    \end{cases}       
\end{equation}

\hspace{15mm}\\
a) Determine the value of k
\\
b) Find P(X $<$ 2),P(X $\leq$ 2),P(X $\geq$ 2)
\\
\\
\textbf{Solution}\\
If we expand the probabilities given further more
by substituting the value of x and only considering
0 to 4 hours as the probability of studying in the
remaining hours is zero, we get\\
\begin{center}
\begin{tabular}{|c|c|c|c|c|c|}
    \hline
    x &  0 & 1 & 2 \\
    \hline
    $\Pr\brak{X=x}$ & k & 2k & 3k\\
    \hline  
\end{tabular}
\end{center}

we also know that,
\begin{align}
    \sum_{k = 0}^2 \Pr\brak{X = k} = 1 \label{eq 2.0.1}
\end{align}

By substituting the probabilities in \eqref{eq 2.0.1}
\begin{align}
& \implies k + 2k + 3k  = 1 \\
& \implies 6k = 1 
\end{align}
\begin{align}
    k = 0.167
\end{align}
\begin{center}
\begin{tabular}{|c|c|c|c|c|c|}
    \hline
    x &  0 & 1 & 2\\
    \hline
    $\Pr\brak{X=x}$ & 0.167 & 0.334 & 0.501\\
    \hline
\end{tabular} 
\end{center}
We know that, Cumulative Distributive Function (CDF) 
\begin{align}
    F(x) = \Pr\brak{X \le x}
\end{align}
\begin{table}[ht]
  
  \centering
  \begin{tabular}{|c|c|c|c|c|c|}
    \hline
    x &  0 & 1 & 2\\
    \hline
    $F(X)$ & 0.167 & 0.501 & 1.00 \\
    \hline
\end{tabular} 
\end{table}
\\And also, 
\begin{align}
     \Pr\brak{x < X \le y} = F\brak{y} - F\brak{x} 
\end{align}
 \begin{enumerate}
        \item P(X $<$ 2)
           \begin{align}
            & \implies \sum_{k = 0}^1 \Pr\brak{X = k} = \Pr\brak{X \ge 2}\\
            & \implies \Pr\brak{0 < X \le 1} 
        \end{align}
        \begin{align}
            & = F(1)\\
            & = 0.501
        \end{align}
        \item P(X $\leq$ 2)
        \begin{align}
         & \implies \sum_{k = 0}^2  \Pr\brak{X = k} = \Pr \brak{X \le 2}
         \end{align}
        \begin{align}
            & = F(2)\\
            & = 1   
        \end{align}
        
        \item P(X $\geq$ 2)
        \begin{align}
            & \implies \Pr\brak{1 < X \le 2} 
        \end{align}
        \begin{align}
            & = F(2) - F(1)\\
            & = 1.002 - 0.501\\
            & = 0.501
        \end{align}
    \end{enumerate}
  
\end{document}