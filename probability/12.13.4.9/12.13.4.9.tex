%\documentclass[class=article, crop=false]{standalone}
\documentclass{article}
\usepackage{amssymb,amsfonts,amsthm,amsmath}
\usepackage{enumitem}
\usepackage{hyperref,xcolor}
\hypersetup{
    colorlinks,
    urlcolor={black}  %black!50!blue
}
\providecommand{\pr}[1]{\ensuremath{\Pr\left(#1\right)}}
\providecommand{\cbrak}[1]{\ensuremath{\left\{#1\right\}}}
\newcommand{\solution}{\noindent \textbf{Solution: }}
%\newcommand{\varsol}{\noindent \textbf{Aliter: }}
\newcommand*{\permcomb}[4][0mu]{{{}^{#3}\mkern#1#2_{#4}}}
%\newcommand*{\perm}[1][-3mu]{\permcomb[#1]{P}}
\newcommand*{\comb}[1][-1mu]{\permcomb[#1]{C}}
\setlist[enumerate]{font=\small\bfseries}
\renewcommand\thefootnote{\textcolor{black}{\arabic{footnote}}}

\begin{document}

\title{PROBABILITY}
\author{\Large DIVYA SAI - FWC22094}
\date{}

\maketitle
\begin{enumerate}
[label=16.\arabic{enumi}.\arabic{enumii}]%,ref=\thesection.\theenumi.\theenumi]
\numberwithin{equation}{enumi}
\setcounter{enumi}{3}
\setcounter{enumii}{10}

\item \footnote{Read question numbers as (CHAPTER NUMBER).(EXERCISE NUMBER).(QUESTION NUMBER)} {The random variable $X$ has a probability distribution \pr{X} of the following form.Where $k$ is some number: }
\begin{align}
  \pr{X} =
    \begin{cases}
      k,  & \text{x=0}\\
      2k, & \text{x=1}\\
      3k, & \text{x=2}\\
      0 , & \text{otherwise}
    \end{cases}       
\end{align}

a) Determine the value of $k$ 

b) Find \pr{X < 2},\pr{X \leq 2},\pr{X \geq 2}  


\solution\\
we know that,
Sum of Probabilities = 1.
\begin{align}
&k + 2k + 3k  = 1 &\\
&6k = 1 &\\
&k = \frac{1}{6}&
\end{align}
Using CDF,
\begin{align}
  F_X(k) =
    \begin{cases}
      0,  & \text{ x$<$0}\\
      \frac{1}{6}, & \text{ 0 $\leq$ x $<$ 1}\\
      \frac{1}{2}, & \text{ 1 $\leq$ x $<$ 2}\\
      1 , & \text{ x $\geq$ 2}
    \end{cases}       
\end{align}

 \begin{enumerate}
        \item \pr{X < 2}
           \begin{align}
              \pr{0 < X \le 1} 
              & = F(1)\\
              & = \frac{1}{2}
        \end{align}
\end{enumerate}
\begin{enumerate}
        \item \pr{X \leq 2}
        \begin{align}
            \pr{X \le 2}
            & = F(2)\\
            & = 1   
        \end{align}
\end{enumerate}
\begin{enumerate}
        \item \pr{X \geq 2}
        \begin{align}
             \pr{X \geq 2}
               & = 1-\pr{X < 2} \\
               & = 1 - F(1)\\
               & = 1 - \frac{1}{2}\\
               & = \frac{1}{2}
        \end{align}
    \end{enumerate}
\end{enumerate}
\end{document}