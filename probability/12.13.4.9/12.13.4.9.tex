%\documentclass[class=article, crop=false]{standalone}
\documentclass{article}
\usepackage{amssymb,amsfonts,amsthm,amsmath}
\usepackage{enumitem}
\usepackage{hyperref,xcolor}
\hypersetup{
    colorlinks,
    urlcolor={black}  %black!50!blue
}

\providecommand{\cbrak}[1]{\ensuremath{\left\{#1\right\}}}
\newcommand{\solution}{\noindent \textbf{Solution: }}
%\newcommand{\varsol}{\noindent \textbf{Aliter: }}
\newcommand*{\permcomb}[4][0mu]{{{}^{#3}\mkern#1#2_{#4}}}
%\newcommand*{\perm}[1][-3mu]{\permcomb[#1]{P}}
\newcommand*{\comb}[1][-1mu]{\permcomb[#1]{C}}
\setlist[enumerate]{font=\small\bfseries}
\renewcommand\thefootnote{\textcolor{black}{\arabic{footnote}}}

\begin{document}

\title{PROBABILITY}
\author{\Large DIVYA SAI - FWC22094}
\date{}

\maketitle
\begin{enumerate}
[label=16.\arabic{enumi}.\arabic{enumii}]%,ref=\thesection.\theenumi.\theenumi]
\numberwithin{equation}{enumi}
\setcounter{enumi}{3}
\setcounter{enumii}{10}

\item \footnote{Read question numbers as (CHAPTER NUMBER).(EXERCISE NUMBER).(QUESTION NUMBER)}\textbf {The random variable X has a probability distribution P(X) of the following form.where k is some number: }
\begin{align}
  P(X) =
    \begin{cases}
      k,  & \text{if x=0}\\
      2k, & \text{if x=1}\\
      3k, & \text{if x=2}\\
      0 , & \text{otherwise}
    \end{cases}       
\end{align}

a) Determine the value of k 

b) Find P(X $<$ 2),P(X $\leq$ 2),P(X $\geq$ 2)  


\textbf{Solution}
If we expand the probabilities given further more
by substituting the value of x and only considering
0 to 4 hours as the probability of studying in the
remaining hours is zero, we get\\
\begin{center}
\begin{tabular}{|c|c|c|c|c|c|}
    \hline
    x &  0 & 1 & 2 \\
    \hline
    $\Pr\cbrak{X=x}$ & k & 2k & 3k\\
    \hline  
\end{tabular}
\end{center}

we also know that,
\begin{align}
    \sum_{k = 0}^2 \Pr\cbrak{X = k} = 1 \label{eq 2.0.1}
\end{align}

By substituting the probabilities in \eqref{eq 2.0.1}
\begin{align}
& \implies k + 2k + 3k  = 1 \\
& \implies 6k = 1 
\end{align}
\begin{align}
    k = 0.167
\end{align}
\begin{center}
\begin{tabular}{|c|c|c|c|c|c|}
    \hline
    x &  0 & 1 & 2\\
    \hline
    $\Pr\cbrak{X=x}$ & 0.167 & 0.334 & 0.501\\
    \hline
\end{tabular} 
\end{center}
We know that, Cumulative Distributive Function (CDF) 
\begin{align}
    F(x) = \Pr\cbrak{X \le x}
\end{align}
\begin{table}[ht]
  
  \centering
  \begin{tabular}{|c|c|c|c|c|c|}
    \hline
    x &  0 & 1 & 2\\
    \hline
    $F(X)$ & 0.167 & 0.501 & 1.00 \\
    \hline
\end{tabular} 
\end{table}
\\And also, 
\begin{align}
     \Pr\cbrak{x < X \le y} = F\cbrak{y} - F\cbrak{x} 
\end{align}
 \begin{enumerate}
        \item P(X $<$ 2)
           \begin{align*}
            & \implies \sum_{k = 0}^1 \Pr\cbrak{X = k} = \Pr\cbrak{X \ge 2}\\
            & \implies \Pr\cbrak{0 < X \le 1} 
        \end{align*}
        \begin{align*}
            & = F(1)\\
            & = 0.501
        \end{align*}
        \item P(X $\leq$ 2)
        \begin{align*}
         & \implies \sum_{k = 0}^2  \Pr\cbrak{X = k} = \Pr \cbrak{X \le 2}
         \end{align*}
        \begin{align*}
            & = F(2)\\
            & = 1   
        \end{align*}
        
        \item P(X $\geq$ 2)
        \begin{align*}
            & \implies \Pr\cbrak{1 < X \le 2} 
        \end{align*}
        \begin{align*}
            & = F(2) - F(1)\\
            & = 1.002 - 0.501\\
            & = 0.501
        \end{align*}
\end{enumerate}
\end{enumerate}
\end{document}