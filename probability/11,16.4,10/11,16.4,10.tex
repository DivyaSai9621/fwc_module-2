\def\mytitle{PROBABILITY}
\def\myauthor{Divya Sai}
\def\contact{nanneboinadivyasai@gmail.com}
\def\mymodule{Future Wireless Communication (FWC)}
\documentclass[10pt, a4paper]{article}
\usepackage[a4paper,outer=1.5cm,inner=1.5cm,top=1.75cm,bottom=1.5cm]{geometry}
\usepackage{graphicx}
\graphicspath{{./images/}}
\usepackage[colorlinks,linkcolor={black},citecolor={blue!80!black},urlcolor={blue!80!black}]{hyperref}
\usepackage[parfill]{parskip}
\usepackage{lmodern}
\usepackage{tikz}
\usepackage{tabularx}
\usepackage{tensor}
\usepackage{amsmath}
\newcommand\Perms[2]{\tensor[^{#2}]P{_{#1}}}
\usepackage{circuitikz}
\usetikzlibrary{calc}
\usepackage{amsmath}
\usepackage{amssymb}
\renewcommand*\familydefault{\sfdefault}
\usepackage{watermark}
\usepackage{lipsum}
\usepackage{xcolor}
\usepackage{listings}
\usepackage{float}
\usepackage{titlesec}
\newcommand{\myvec}[1]{\ensuremath{\begin{pmatrix}#1\end{pmatrix}}}
\newcommand{\mydet}[1]{\ensuremath{\begin{vmatrix}#1\end{vmatrix}}}
\providecommand{\brak}[1]{\ensuremath{\left(#1\right)}}
\providecommand{\lbrak}[1]{\ensuremath{\left(#1\right.}}
\providecommand{\rbrak}[1]{\ensuremath{\left.#1\right)}}
\providecommand{\sbrak}[1]{\ensuremath{{}\left[#1\right]}}
\providecommand{\mtx}[1]{\mathbf{#1}}
\titlespacing{\subsection}{1pt}{\parskip}{3pt}
\titlespacing{\subsubsection}{0pt}{\parskip}{-\parskip}
\titlespacing{\paragraph}{0pt}{\parskip}{\parskip}
\newcommand{\figuremacro}[5]{
    \begin{figure}[#1]
        \centering
   
        \includegraphics[width=#5\columnwidth]{#2}
        \caption[#3]{\textbf{#3}#4}
        \label{fig:#2}
    \end{figure}
}
\let\vec\mathbf
\lstset{
frame=single, 
breaklines=true,
columns=fullflexible
}

\title{\mytitle}
\author{\myauthor\hspace{1em} \\\contact\\FWC22094\hspace{6.5em}IITH\hspace{0.5em}\mymodule\hspace{6em}Module 2}
\date{}
\begin{document}
\maketitle
\paragraph*{\large Q-11,16.4,10}
\paragraph*{\large The number lock of a suitcase has 4 wheels each labelled with ten digits i.e. from 0 to 9.The lock opens with a sequence of four digits with no repeats.What is the probability of a person getting the right sequence to open the suitcase?}
\hspace{160mm}\\
\textbf{Solution:}\\
Let,the numbers be $X =\lbrace{0,1,2,3}\rbrace$ and the digits be $Y=\lbrace{0,1,2...9}\rbrace$\\
There are 10 digits out of which 4 digits are to be chosen with no repeats\\
possible placement of digits are,
\begin{equation}
^nP_r=\frac{n!}{(n-r)!}
\end{equation}
\\n=10 ,r= 4
\begin{equation}
^{10}P_4=\frac{10!}{(10-4)!}=\frac{10!}{6!}=\frac{10 \times 9 \times 8 \times 7 \times	 6!}{6!}=5040
\end{equation}
But since, the lock can open with only one correct sequence out of the all 4 digit numbers.
\\Let A be the event the correct sequence is selected So,n(A)=1
\\Probability of correct sequence is selected,P(A)
\\
\begin{align}
=\frac{\text{No of Possible outcomes}}{\text{Total number of outcomes}}
\end{align}\\
\begin{equation}
P(A)=\frac{n(A)}{n(s)}=\frac{1}{5040}
\end{equation}
Therefore, the probability of getting the right sequence to open the suitcase is 
\begin{equation}
\boxed{\frac{1}{5040}}
\end{equation}
\end{document}